\documentclass[a4paper,11pt]{article}
\usepackage{nopageno} % visto che in questo caso abbiamo una pagina sola
\usepackage{lmodern}
\renewcommand*\familydefault{\sfdefault}
\usepackage{sfmath}
%\usepackage{amsmath}
\usepackage[utf8]{inputenc}
\usepackage[T1]{fontenc}
\usepackage[italian]{babel}
\usepackage{indentfirst}
\usepackage{graphicx}
\usepackage{tikz}
\usetikzlibrary{arrows.meta}
\usetikzlibrary{positioning,fit}%calc e positioning permettono il posizionamento avanzato
\usetikzlibrary{decorations,decorations.pathmorphing,decorations.pathreplacing,decorations.markings,decorations.shapes}
\usetikzlibrary{arrows}%se si usano le frecce diverse
\usetikzlibrary{fit,backgrounds}


\usepackage{wrapfig}
\usepackage{enumitem}
% \usepackage[group-separator={\,}]{siunitx}
\usepackage[left=2cm, right=2cm, bottom=3cm]{geometry}
\frenchspacing

\newcommand{\num}[1]{#1}

% Macro varie...
\newcommand{\file}[1]{\texttt{#1}}
\renewcommand{\arraystretch}{1.3}
\newcommand{\esempio}[2]{
\noindent\begin{minipage}{\textwidth}
\begin{tabular}{|p{11cm}|p{5cm}|}
	\hline
        \textbf{\file{input (da stdin)}} & \textbf{\file{output (su stdout)}}\\
%	\textbf{File \file{input.txt}} & \textbf{File \file{output.txt}}\\
	\hline
	\tt \small #1 &
	\tt \small #2 \\
	\hline
\end{tabular}
\end{minipage}
}

\newcommand{\sezionetesto}[1]{
    \section*{#1}
}


%%%%% I seguenti campi verranno sovrascritti dall'\include{nomebreve} %%%%%
\newcommand{\nomebreve}{}
\newcommand{\titolo}{}

% Modificare a proprio piacimento:
\newcommand{\introduzione}{
%    \noindent{\Large \gara{}}
%    \vspace{0.5cm}
    \noindent{\Huge \textbf \titolo{}~(\texttt{\nomebreve{}})}
    \vspace{0.2cm}\\
}

\begin{document}

\renewcommand{\nomebreve}{conta\_inversioni}
\renewcommand{\titolo}{Numero di Inversioni}

\introduzione{}

Devi scrivere una funzione {\tt numero\_di\_inversioni} che riceve in input una lista $p$ di numeri naturali tutti distinti e ritorna il numero di inversioni di $p$. Un'inversione di $p$ è una qualsiasi coppia di indici $(i,j)$ con $0\leq i < j < len(p)$ tale che $p[i] > p[j]$.

Trovi un template della soluzione nel file \textbf{\file{conta\_inversioni\_template\_sol.py}}, dovrai solo risistemare l'implementazione della funzione {\tt num\_invertions} che attualmente non fà quanto richiesto: 

\begin{verbatim}
def numero_di_inversioni(p):
    return 42
\end{verbatim}


\sezionetesto{Dati di input}
Il vostro programma riceve in input una lista di numeri naturali separati da spazi e tutti diversi tra di loro.

\sezionetesto{Dati di output}

Il programma deve ritornare in output il numero di inversioni per la lista ricevuta in input.

% Esempi
\sezionetesto{Esempi di input/output}
\esempio{1 0 3 2}{2}

\esempio{0 1 2 3 4}{0}

\esempio{4 3 2 1 0}{10}

% Assunzioni
\sezionetesto{Assunzioni}
\begin{itemize}[nolistsep, noitemsep]
  \item $1 \le n \le 100\,000$.
\end{itemize}

  \section*{Subtask}
  \begin{itemize}
    \item \textbf{Subtask 1 [0 punti]:} gli esempi del testo.
    \item \textbf{Subtask 2 [15 punti]:} $n=2$.
    \item \textbf{Subtask 3 [15 punti]:} $n=3$.
    \item \textbf{Subtask 4 [30 punti]:} $n\leq 101$ e la lista è ordinata (in ordine crescente o decrescente).
    \item \textbf{Subtask 5 [40 punti]:} $n \leq 100$.
    \item \textbf{Subtask 6 [0 punti]:} $n \leq 100\,000$.
  \end{itemize}
  


\end{document}
