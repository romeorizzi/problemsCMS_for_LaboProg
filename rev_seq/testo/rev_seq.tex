\renewcommand{\nomebreve}{rovesciaseq}
\renewcommand{\titolo}{Rovesciare una sequenza}

\introduzione{}

Ricevete in input una sequenza di $N$ numeri interi.
In output dovete riproporre gli stessi $N$ numeri,
ma in ordine inverso.


\sezionetesto{Dati di input}
La prima riga del file \verb'input.txt' contiene un numero intero e positivo $N$, la lunghezza della sequenza da rovoesciare.
La seconda riga contiene, in ordine, gli $N$ numeri della sequenza.

\sezionetesto{Dati di output}
La prima riga del file \verb'output.txt' contiene l'intero $N$.
La seconda riga contiene, in ordine, gli $N$ numeri della sequenza rovesciata.


% Esempi
\sezionetesto{Esempio di input/output}
\esempio{
6

1 -2 -3 4 -5 9
}{
6

9 -5 4 -3 -2 1}
\esempio{
5

15 24 -33 -42 51
}{
5

51 -42 -33 24 15}

% Assunzioni
\sezionetesto{Assunzioni e note}
\begin{itemize}[nolistsep, noitemsep]
  \item $2 \le N \le 1\,000\,000$.
\end{itemize}
  
  \section*{Subtask}
  \begin{itemize}
    \item \textbf{Subtask 0 [10 punti]:} i due esempi del testo.
    \item \textbf{Subtask 1 [10 punti]:} $N = 3$.
    \item \textbf{Subtask 2 [20 punti]:} $N \leq 5$.
    \item \textbf{Subtask 3 [20 punti]:} $N \leq 20$.
    \item \textbf{Subtask 4 [40 punti]:} nessuna restrizione.
  \end{itemize}
  
