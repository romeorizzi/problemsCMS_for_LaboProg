\documentclass[a4paper,11pt]{article}
\usepackage{nopageno} % visto che in questo caso abbiamo una pagina sola
\usepackage{lmodern}
\renewcommand*\familydefault{\sfdefault}
\usepackage{sfmath}
%\usepackage{amsmath}
\usepackage[utf8]{inputenc}
\usepackage[T1]{fontenc}
\usepackage[italian]{babel}
\usepackage{indentfirst}
\usepackage{graphicx}
\usepackage{tikz}
\usepackage{wrapfig}
\newcommand*\circled[1]{\tikz[baseline=(char.base)]{
		\node[shape=circle,draw,inner sep=2pt] (char) {#1};}}
\usepackage{enumitem}
% \usepackage[group-separator={\,}]{siunitx}
\usepackage[left=2cm, right=2cm, bottom=3cm]{geometry}
\frenchspacing

\newcommand{\num}[1]{#1}

% Macro varie...
\newcommand{\file}[1]{\texttt{#1}}
\renewcommand{\arraystretch}{1.3}
\newcommand{\esempio}[2]{
\noindent\begin{minipage}{\textwidth}
\begin{tabular}{|p{11cm}|p{5cm}|}
	\hline
        \textbf{\file{input (da stdin)}} & \textbf{\file{output (su stdout)}}\\
%	\textbf{File \file{input.txt}} & \textbf{File \file{output.txt}}\\
	\hline
	\tt \small #1 &
	\tt \small #2 \\
	\hline
\end{tabular}
\end{minipage}
}

\newcommand{\sezionetesto}[1]{
    \section*{#1}
}


%%%%% I seguenti campi verranno sovrascritti dall'\include{nomebreve} %%%%%
\newcommand{\nomebreve}{}
\newcommand{\titolo}{}

% Modificare a proprio piacimento:
\newcommand{\introduzione}{
%    \noindent{\Large \gara{}}
%    \vspace{0.5cm}
    \noindent{\Huge \textbf \titolo{}~(\texttt{\nomebreve{}})}
    \vspace{0.2cm}\\
}

\begin{document}

\renewcommand{\nomebreve}{list\_divisors}
\renewcommand{\titolo}{Elencare i divisori di un numero}

\introduzione{}

Ricevuto in input da stdin un numero naturale positivo $n$,
il tuo programma dovrà elencare tutti quei numeri naturali che dividano $n$ senza produrre alcun resto. Tali numeri vanno scritti su stdout, collocati uno per riga, ed in ordine crescente.

% Esempi
\sezionetesto{Esempio di input/output}
\esempio{5}{
1

5
}

\sezionetesto{Esempio di input/output}
\esempio{6}{
1

2

3

6
}


% Assunzioni
\sezionetesto{Assunzioni e note}
\begin{itemize}[nolistsep, noitemsep]
  \item $1 \le n \le 1\,000\,000$.
\end{itemize}
  
  \section*{Subtask}
  \begin{itemize}
    \item \textbf{Subtask 1 [20 punti]:} gli esempi del testo.
    \item \textbf{Subtask 2 [20 punti]:} $n\leq 10$.
    \item \textbf{Subtask 3 [20 punti]:} $n\leq 100$.
    \item \textbf{Subtask 4 [20 punti]:} $n\leq 1000$.
    \item \textbf{Subtask 5 [20 punti]:} $n\leq 1\,000\,000$.
  \end{itemize}
  


\end{document}
