\renewcommand{\nomebreve}{list\_divisors}
\renewcommand{\titolo}{Elencare i divisori di un numero}

\introduzione{}

Ricevuto in input da stdin un numero naturale positivo $n$,
il tuo programma dovrà elencare tutti quei numeri naturali che dividano $n$ senza produrre alcun resto. Tali numeri vanno scritti su stdout, collocati uno per riga, ed in ordine crescente.

% Esempi
\sezionetesto{Esempio di input/output}
\esempio{5}{
1

5
}

\sezionetesto{Esempio di input/output}
\esempio{6}{
1

2

3

6
}


% Assunzioni
\sezionetesto{Assunzioni e note}
\begin{itemize}[nolistsep, noitemsep]
  \item $1 \le n \le 1\,000\,000$.
\end{itemize}
  
  \section*{Subtask}
  \begin{itemize}
    \item \textbf{Subtask 1 [20 punti]:} gli esempi del testo.
    \item \textbf{Subtask 2 [20 punti]:} $n\leq 10$.
    \item \textbf{Subtask 3 [20 punti]:} $n\leq 100$.
    \item \textbf{Subtask 4 [20 punti]:} $n\leq 1000$.
    \item \textbf{Subtask 5 [20 punti]:} $n\leq 1\,000\,000$.
  \end{itemize}
  
