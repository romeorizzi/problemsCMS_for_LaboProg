\renewcommand{\nomebreve}{bisestile\_giuliano}
\renewcommand{\titolo}{Esercizio uso if: Riconscimento di anni bisestili}

\introduzione{}

Un anno bisestile è un anno solare di 366 giorni piuttosto che di 365.
Il giorno in più viene inserito nel mese di febbraio, il più corto dell'anno, che negli anni bisestili arriva a contare 29 giorni anziché 28.
Serve come accorgimento utilizzato in quasi tutti i calendari solari (quali quelli giuliano e gregoriano) per evitare lo slittamento delle stagioni.
In questo modo si può ottenere una durata media dell'anno pari a un numero non intero di giorni. 
Nel calendario giuliano è bisestile un anno ogni 4 (quelli la cui numerazione è divisibile per $4$). La durata media dell'anno diventa così di $365,25$ giorni ($365$ giorni e $6$ ore) e la differenza rispetto all'anno tropico si riduce da $5,8128$ ore in difetto ad appena $11$ minuti e $14$ secondi in eccesso.

\sezionetesto{Dati di input}
Il vostro programma riceve come suo input un numero naturale $N$; esso indica un qualche anno solare a partire da quello della nascita di Cristo (l'anno zero).

\sezionetesto{Dati di output}

Dovete ritornare in output il numero di giorni di cui si compone l'anno $N$ di Nostro Signore.

% Esempi
\sezionetesto{Esempio di input/output}
\esempio{0}{
366
}

\sezionetesto{Esempio di input/output}
\esempio{1}{
365
}

\sezionetesto{Esempio di input/output}
\esempio{100}{
366
}

\sezionetesto{Esempio di input/output}
\esempio{2018}{
365
}


% Assunzioni
\sezionetesto{Assunzioni e note}
\begin{itemize}[nolistsep, noitemsep]
  \item $0 \le N \le 1\,000\,000$.
\end{itemize}
  
  \section*{Subtask}
  \begin{itemize}
    \item \textbf{Subtask 1 [20 punti]:} gli esempi del testo.
    \item \textbf{Subtask 2 [80 punti]:} nessuna restrizione.
  \end{itemize}
  
