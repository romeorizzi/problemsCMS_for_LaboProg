\documentclass[a4paper,11pt]{article}
\usepackage{nopageno} % visto che in questo caso abbiamo una pagina sola
\usepackage{lmodern}
\renewcommand*\familydefault{\sfdefault}
\usepackage{sfmath}
\usepackage[utf8]{inputenc}
\usepackage[T1]{fontenc}
\usepackage[italian]{babel}
\usepackage{indentfirst}
\usepackage{graphicx}
\usepackage{tikz}
\usepackage{wrapfig}
\newcommand*\circled[1]{\tikz[baseline=(char.base)]{
		\node[shape=circle,draw,inner sep=2pt] (char) {#1};}}
\usepackage{enumitem}
% \usepackage[group-separator={\,}]{siunitx}
\usepackage[left=2cm, right=2cm, bottom=3cm]{geometry}
\frenchspacing

\newcommand{\num}[1]{#1}

% Macro varie...
\newcommand{\file}[1]{\texttt{#1}}
\renewcommand{\arraystretch}{1.3}
\newcommand{\esempio}[2]{
\noindent\begin{minipage}{\textwidth}
\begin{tabular}{|p{11cm}|p{5cm}|}
	\hline
	\textbf{File \file{input.txt}} & \textbf{File \file{output.txt}}\\
	\hline
	\tt \small #1 &
	\tt \small #2 \\
	\hline
\end{tabular}
\end{minipage}
}

\newcommand{\sezionetesto}[1]{
    \section*{#1}
}

\newcommand{\gara}{Corso Tandem 2015 - prima edizione}

%%%%% I seguenti campi verranno sovrascritti dall'\include{nomebreve} %%%%%
\newcommand{\nomebreve}{}
\newcommand{\titolo}{}

% Modificare a proprio piacimento:
\newcommand{\introduzione}{
%    \noindent{\Large \gara{}}
%    \vspace{0.5cm}
    \noindent{\Huge \textbf \titolo{}~(\texttt{\nomebreve{}})}
    \vspace{0.2cm}\\
}

\begin{document}

\renewcommand{\nomebreve}{max\_seq}
\renewcommand{\titolo}{Find the maximum element in a sequence}

\introduzione{}

You receive in input a sequence of $N$ integer numbers.
Return in output the maximum among them.

\sezionetesto{input description}
The first line of the file \verb'input.txt'
contains a positive integer number $N$.
The second and last line
contains a sequence of $N$ integer numbers separated by spaces.

\sezionetesto{output description}
In the file \verb'output.txt' write one single line containing only one number: the maximum integer value contained in the input sequence.


% Esempi
\sezionetesto{input/output example}
\esempio{
6

1 -2 -3 4 -5 9
}{9}
\esempio{
5

15 24 -33 -42 21
}{24}

% Assunzioni
\sezionetesto{Assumptions}
\begin{itemize}[nolistsep, noitemsep]
  \item $2 \le N \le 1\,000\,000$.
\end{itemize}
  
  \section*{Subtasks}
  \begin{itemize}
    \item \textbf{Subtask 1 [10 punti]:} the two examples above.
    \item \textbf{Subtask 2 [10 punti]:} $N = 2$.
    \item \textbf{Subtask 3 [20 punti]:} $N \leq 5$.
    \item \textbf{Subtask 4 [20 punti]:} $N \leq 20$.
    \item \textbf{Subtask 5 [40 punti]:} no special restriction.
  \end{itemize}
  


\end{document}
