\renewcommand{\nomebreve}{max\_seq}
\renewcommand{\titolo}{Find the maximum element in a sequence}

\introduzione{}

You receive in input a sequence of $N$ integer numbers.
Return in output the maximum among them.

\sezionetesto{input description}
The first line of the file \verb'input.txt'
contains a positive integer number $N$.
The second and last line
contains a sequence of $N$ integer numbers separated by spaces.

\sezionetesto{output description}
In the file \verb'output.txt' write one single line containing only one number: the maximum integer value contained in the input sequence.


% Esempi
\sezionetesto{input/output example}
\esempio{
6

1 -2 -3 4 -5 9
}{9}
\esempio{
5

15 24 -33 -42 21
}{24}

% Assunzioni
\sezionetesto{Assumptions}
\begin{itemize}[nolistsep, noitemsep]
  \item $2 \le N \le 1\,000\,000$.
\end{itemize}
  
  \section*{Subtasks}
  \begin{itemize}
    \item \textbf{Subtask 1 [10 punti]:} the two examples above.
    \item \textbf{Subtask 2 [10 punti]:} $N = 2$.
    \item \textbf{Subtask 3 [20 punti]:} $N \leq 5$.
    \item \textbf{Subtask 4 [20 punti]:} $N \leq 20$.
    \item \textbf{Subtask 5 [40 punti]:} no special restriction.
  \end{itemize}
  
