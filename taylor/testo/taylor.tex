\renewcommand{\nomebreve}{taylor}
\renewcommand{\titolo}{Computo approssimato di funzioni tramite Taylor e Mc-Laurin}

\newcommand{\apx}{{{apx}}}

\introduzione{}

Molti sistemi di computo si avvalgono delle serie di Taylor e Mc-Laurin per computare approssimazioni successive del valore di funzioni indefinitamente derivabili.

Ad esempio, si potrebbe avvalersi delle seguenti scritture:

\begin{itemize}
\item[1] $e^x = \sum_{i=0}^{\infty} \frac{[e^x]_{[x=0]}x^i}{i!} = \sum_{i=0}^{\infty} \frac{x^i}{i!}$, da cui l'aprossimazione di rango $k$ è $\apx(e^x;k) := \sum_{i=1}^{k} \frac{x^i}{i!}$;

\item[2] $\sin x = \sum_{i=0}^{\infty} (-1)^i \frac{x^{(2i+1)}}{(2i+1)!}$,
  da cui $\apx(\sin x;k) := \sum_{i=0}^{k} (-1)^i \frac{x^{(2i+1)}}{(2i+1)!}$;  

\item[3] $\log (1+x) = \sum_{i=0}^{\infty} (-1)^i \frac{x^i}{i}$,
  da cui $\apx(\log (1+x);k) := \sum_{i=0}^{k} (-1)^i \frac{x^i}{i}$.   
\end{itemize}

Uno alla volta (uno per riga),
il vostro programma riceve in input i seguenti $4$ valori:

\begin{itemize}
\item[1] un valore intero $t\in \{1,2,3\}$ atto a specificare quali delle tre formule sopra si intenda mettere alla prova: quella per il computo di $f_1(x)=e^x$,
  quella per il computo di $f_2(x)=\sin x$, oppure quella per il computo di $f_3(x) = \log (1+x)$;
\item[2] un valore float da assegnare alla variabile indipendente $x$, per il coputo approssimato di $f_t(x)$; 
\item[3] un valore float $s$ che offre una stima molto precisa di $f_t(x)$;
\item[4] un valore reale $\varepsilon$ che esprima una tolleranza;
\end{itemize}

Compito del vostro programma è il computo del più piccolo intero $k$
tale che $|s-\apx(f_t(x);k)| < \varepsilon$.

\sezionetesto{Dati di input}
Il vostro programma riceve come suo input i valori $t, x, s, \varepsilon$
in questo preciso ordine. Li legge da stdin uno alla volta, il primo è un numenro naturale e gli altri sono dei float.

\sezionetesto{Dati di output}

Dovete ritornare in output il numero naturale $k$ più piccolo tale che l'aprossimazione di rango $k$ possa bastare ad approssimare $s$ a meno di $\varepsilon$.

% Esempi
\sezionetesto{Esempio di input/output}
\esempio{
1
  
1.0

2.71828182845904523536

0.0001
}{
?  
}

\sezionetesto{Esempio di input/output}
\esempio{
2
  
0.0

0.0

0.693147181

0.00001
}{
0
}

\sezionetesto{Esempio di input/output}
\esempio{
3

1.0

0.00001
}{
  ?
}



% Assunzioni
%\sezionetesto{Assunzioni e note}
%\begin{itemize}[nolistsep, noitemsep]
%  \item $0 \le N \le 1\,000\,000$.
%\end{itemize}
  
  \section*{Subtask}
  \begin{itemize}
    \item \textbf{Subtask 1 [25 punti]:} gli esempi del testo.
    \item \textbf{Subtask 2 [25 punti]:} $t=1$.
    \item \textbf{Subtask 3 [25 punti]:} $t=2$.
    \item \textbf{Subtask 4 [25 punti]:} $t=3$.
  \end{itemize}
  
