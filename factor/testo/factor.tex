\renewcommand{\nomebreve}{factor}
\renewcommand{\titolo}{Produrre la fattorizzazione di un naturale come prodotto di primi}

\introduzione{}

Un numero naturale $n$, con $n\geq 2$, si dice \emph{primo} se ammette solamente i segunti due divisori ``benali'', comuni a tutti i numeri naturali:
\begin{itemize}
\item lo uno divide ogni numero naturale; lo uno non viene considerato un numero primo, e, proprio per meglio distinguerlo, in questo contesto come in sue generalizzazioni, viene spesso chiamato l'\emph{unità}.
\item ogni numero naturale, forse con la sola eccezione dello zero, divide splendidamente sè stesso.  
\end{itemize}

La ragione di non considerare lo uno alla stregua di un numero primo è per non mandare a rotoli il teorema fondamentale dei numeri (Euclide):

\begin{quote}
  Ogni numero naturale può essere scritto come prodotto di numeri interi.
  E tale scrittura è unica.
\end{quote}

Ricevuto in input da stdin un numero naturale positivo $n$,
il tuo programma dovrà scrivere su stdout tale scrittura unica.
Chiediamo che i fattori primi del numero vengano listati in ordine di granzezza, così che la stessa stringa che sei chiamato a scrivere sia di fatto univocamente determinata dall'input.

% Esempi
\sezionetesto{Esempio di input/output}
\esempio{5}{
5
}

\sezionetesto{Esempio di input/output}
\esempio{6}{
2*3*6
}

\sezionetesto{Esempio di input/output}
\esempio{8}{
2*2*2
}

\sezionetesto{Esempio di input/output}
\esempio{12}{
2*2*3
}


% Assunzioni
\sezionetesto{Assunzioni e note}
\begin{itemize}[nolistsep, noitemsep]
  \item $1 \le n \le 1\,000\,000$.
\end{itemize}
  
  \section*{Subtask}
  \begin{itemize}
    \item \textbf{Subtask 1 [20 punti]:} gli esempi del testo.
    \item \textbf{Subtask 2 [20 punti]:} $n\leq 20$.
    \item \textbf{Subtask 3 [20 punti]:} $n\leq 100$.
    \item \textbf{Subtask 4 [20 punti]:} $n\leq 1000$.
    \item \textbf{Subtask 5 [20 punti]:} $n\leq 1\,000\,000$.
  \end{itemize}
  
