\renewcommand{\nomebreve}{for2\_file}
\renewcommand{\titolo}{Esercizio sui for annidati - computo tabelline}

\introduzione{}

Ricevete in input un numero naturale positivo $N$ e,
in output, dovete generare la tabellina per $N$,
ossia una tabella di $N\times N$ numeri, dove,
nella riga $i$-esima (quella relativa ai prodotti del numero $i$)
e colonna $j$-esima  (quella relativa ai prodotti del numero $j$),
compaia il valore $i\cdot j$.

\sezionetesto{Dati di input}
La prima ed unica riga del file \verb'input.txt' contiene un numero intero e positivo $N$.

\sezionetesto{Dati di output}

Il file \verb'output.txt' si presenta come la sequente tabella $N\times N$:\\

\begin{tabular}{cccccc}
    $1$   &   $2$    &  $3$     & $\cdots$ &  $N-1$     &  $N$ \\
    $2$   &   $4$    &  $6$     & $\cdots$ & $2N-2$     & $2N$ \\
 $\vdots$ & $\vdots$ & $\ddots$ &          &   $\vdots$ &  $\vdots$ \\
 $\vdots$ & $\vdots$ &          & $\ddots$ &   $\vdots$ &  $\vdots$ \\
  $N-1$   & $2N-2$   & $3N-3$   & $\cdots$ & $N^2-2N+1$ & $N^2-N$ \\
    $N$   &  $2N$    & $3N$     & $\cdots$ &    $N^2-N$ &  $N^2$
\end{tabular}

% Esempi
\sezionetesto{Esempio di input/output}
\esempio{4}{
1 2 3 4

2 4 6 8

3 6 9 12

4 8 12 16}

% Assunzioni
\sezionetesto{Assunzioni e note}
\begin{itemize}[nolistsep, noitemsep]
  \item $1 \le N \le 1\,000$.
\end{itemize}
  
  \section*{Subtask}
  \begin{itemize}
    \item \textbf{Subtask 0 [10 punti]:} l'esempio del testo.
    \item \textbf{Subtask 1 [15 punti]:} $N = 11$.
    \item \textbf{Subtask 2 [25 punti]:} $N \leq 13$.
    \item \textbf{Subtask 4 [50 punti]:} nessuna restrizione.
  \end{itemize}
  
