\renewcommand{\nomebreve}{conta\_multipli}
\renewcommand{\titolo}{Conta i multipli di a non divisibili per b}

\introduzione{}

Devi scrivere una funzione {\tt conta\_multipli} che riceve in input tre numeri naturali $a$, $b$ e $c$ e ritorna il numero di quei numeri interi positivi $n$ tali che valgano le seguenti 3 proprietà:
\begin{itemize}
   \item $n\leq c$;
   \item $a$ divide $n$;
   \item $b$ non divide $n$.
\end{itemize}

Trovi un template della soluzione nel file \textbf{\file{conta\_multipli\_template\_sol.py}}, dovrai solo risistemare l'implementazione della funzione {\tt conta\_multipli} che attualmente non fà quanto richiesto: 

\begin{verbatim}
def conta_multipli(a,b,c):
    return 42
\end{verbatim}


\sezionetesto{Dati di input}
Il vostro programma riceve in input tre numeri naturali separati da spazi.

\sezionetesto{Dati di output}

Il programma deve ritornare il numero di multipli di $a$ non divisibili per $b$ che non eccedano $c$.

% Esempi
\sezionetesto{Esempi di input/output}
\esempio{3 7 21}{6}

\esempio{1 5 100}{80}

\esempio{100 16 100}{1}

% Assunzioni
\sezionetesto{Assunzioni}
\begin{itemize}[nolistsep, noitemsep]
  \item $1 \le a,b,c \le 10^{100}$.
\end{itemize}

  \section*{Subtask}
  \begin{itemize}
    \item \textbf{Subtask 1 [0 punti]:} gli esempi del testo.
    \item \textbf{Subtask 2 [20 punti]:} $a=c$, $\;a,b,c\leq 1000$.
    \item \textbf{Subtask 3 [20 punti]:} $a=1$, $b>c$, $\;a,b,c\leq 1000$.
    \item \textbf{Subtask 4 [20 punti]:} $a=1$, $\;a,b,c\leq 1000$.
    \item \textbf{Subtask 5 [20 punti]:} $b>c$, $\;a,b,c\leq 1000$.
    \item \textbf{Subtask 6 [20 punti]:} $a,b,c \leq 1000$.
    \item \textbf{Subtask 7 [0 punti]:} $a,b,c \leq 10^{100}$.
  \end{itemize}
  
